\documentclass[a4paper, 12pt]{report}
\usepackage{amsmath, amsthm, amsfonts} % depended by \DeclareMathOperator, \newtheorem, \proof, \mathbb
\usepackage[utf8]{inputenc} % utf8 encoding
\usepackage{listings} % for lst in color box
\usepackage[most]{tcolorbox} % for messageshell colorbox
\usepackage{algorithm} % depended by algpseudocode

\usepackage{hyperref} % add clickable link to tableofcontents
\usepackage{enumitem} % use enumerate
\usepackage{float} % for [H] label in figures
\usepackage{url} % for citing wiki

\usepackage{fancyhdr} % add header and footer
\pagestyle{fancy}
\fancyhf{}
\rhead{\leftmark}
\cfoot{\thepage}

\graphicspath{{img/}}

\usepackage{geometry} % set margin
\geometry{left=3cm, right=2.5cm, top=2.5cm, bottom=2.5cm}

\newtheorem{theorem}{Theorem}[section]
\newtheorem{corollary}{Corollary}[theorem]
\newtheorem{lemma}[theorem]{Lemma}
\lstset{
  basicstyle=\ttfamily,
  columns=fullflexible,
  breaklines=true,
}
\newtcblisting{commandshell}{colback=white,colupper=black,colframe=black!75!black,
listing only,listing options={language=sh, breaklines=true,aboveskip=0pt, belowskip=0pt},
every listing line={\small\ttfamily\bfseries{[oracle@tp-shchai]\$} }}

\lstset{language=SQL,morekeywords={PREFIX,java,rdf,rdfs,url}}

\newtcblisting{messageshell}{colback=white,colupper=black,colframe=black!75!black,
listing only,listing options={language={}, basicstyle=\small\ttfamily, breaklines=true,aboveskip=0pt, belowskip=0pt},
every listing line={}}

\title{Semantics Web Lab \#1 and \#2 Report}
\date{}
\author{Shitong CHAI}

\begin{document}

\maketitle
\tableofcontents

\chapter{Lab \#1}
\section{Exercise 2}
\begin{enumerate}
\item Execute the queries qsample1.sparql and qsample2.sparql and explain what is computed by each query.

For the first query, execute the following command:
\begin{commandshell}
cd /opt/semantic-web-lab/myqueries
arq -data=../mydatasets/countries.rdf -query=qsample1.rq.rtf
\end{commandshell}
The result:
\begin{messageshell}
-------------------------------------------------
| x                                             |
=================================================
| <http://telegraphis.net/data/countries/FR#FR> |
-------------------------------------------------
\end{messageshell}
The query selects all the countries which have the ISO Alpha2 index as "FR".
For the second query, execute the following command:
\begin{commandshell}
cd /opt/semantic-web-lab/myqueries
arq -data=../mydatasets/countries.rdf -query=qsample2.rq.rtf
\end{commandshell}
The result:
\begin{messageshell}
---------------------------------------------------------------
| x                                             | y           |
===============================================================
| <http://telegraphis.net/data/countries/FR#FR> | "FRANCE"@en |
| <http://telegraphis.net/data/countries/FR#FR> | "FRANCE"@fr |
---------------------------------------------------------------
\end{messageshell}
The query selects all the countries and the its short name which have the ISO Alpha2 index as "FR".

\item[(a)] What is the current population of France ? 

\begin{lstlisting}[frame=single]
PREFIX countries:<http://telegraphis.net/data/countries>
PREFIX  gn:<http://www.geonames.org/ontology#>
PREFIX geographis:<http://telegraphis.net/ontology/geography/geography#>

Select ?x
where {
?fr geographis:isoAlpha2 "FR" .
<http://telegraphis.net/data/countries/FR#FR> gn:population ?x .
}
\end{lstlisting}
\item[(e)] List all the countries. 
\begin{lstlisting}[frame=single]
PREFIX countries: <http://telegraphis.net/data/countries>
PREFIX  gn:  <http://www.geonames.org/ontology#>
PREFIX geographis:  <http://telegraphis.net/ontology/geography/geography#>

Select distinct ?name
where {
        ?country geographis:isoShortName ?name .

}
\end{lstlisting}

\item[(f)] What is the landArea of France ? How do you interpret the obtained result ? 
\begin{lstlisting}[frame=single]
PREFIX countries: <http://telegraphis.net/data/countries>
PREFIX  gn:  <http://www.geonames.org/ontology#>
PREFIX geographis:  <http://telegraphis.net/ontology/geography/geography#>
PREFIX measurement: <http://telegraphis.net/ontology/measurement/measurement#>
PREFIX rdf: <http://www.w3.org/1999/02/22-rdf-syntax-ns#>

Select ?z ?unit
where {
        ?fr geographis:isoAlpha2 "FR" .
        ?fr geographis:landArea _:a .
        _:a rdf:value ?z .
        _:a measurement:unit ?unit .
}

\end{lstlisting}

\item[(i)] Is there an African country with an ISO 3166-1 numeric code equal to "024" ?
\begin{lstlisting}[frame=single]
PREFIX countries: <http://telegraphis.net/data/countries>
PREFIX  gn:  <http://www.geonames.org/ontology#>
PREFIX geographis:  <http://telegraphis.net/ontology/geography/geography#>

ask
{
        ?c geographis:onContinent <http://telegraphis.net/data/continents/AF#AF>.
        ?c geographis:isoNumeric "024" .

}

\end{lstlisting}

\item[(k)] What is the english name of the country having an ISO 3166-1 numeric code equal to "020" ?
\begin{lstlisting}[frame=single]
PREFIX countries: <http://telegraphis.net/data/countries>
PREFIX  gn:  <http://www.geonames.org/ontology#>
PREFIX geographis:  <http://telegraphis.net/ontology/geography/geography#>

Select ?x
where {
        ?c gn:name ?x .
        ?c geographis:isoNumeric "020" .
}
\end{lstlisting}
\end{enumerate}

\chapter{Lab \#2}
For all the SPARQL codes in the following, they all have the same prefix, which is:
\begin{lstlisting}[frame=single]
PREFIX dbo:<http://dbpedia.org/ontology/>
PREFIX dbr:<http://dbpedia.org/resource/>
prefix foaf:  <http://xmlns.com/foaf/0.1/>
prefix rdf:  <http://www.w3.org/1999/02/22-rdf-syntax-ns#>
PREFIX dct:    <http://purl.org/dc/terms/>
prefix rdfs:<http://www.w3.org/2000/01/rdf-schema#>
\end{lstlisting}

\begin{enumerate}
\item Find all the list of predicates and objects linked to the subject Alan Turing in dbpedia 
\begin{lstlisting}[frame=single]
select distinct ?x 
{
 {
   dbr:Alan_Turing ?y ?x .
 }
 union
 {
   ?x ?y  dbr:Alan_Turing.
 }
} 
\end{lstlisting}


\item Find the homepage of Alan Turing
\begin{lstlisting}[frame=single]
select distinct ?y
{

   ?r foaf:name "Alan Turing"@en .
   ?r foaf:isPrimaryTopicOf ?y .

} 
\end{lstlisting}

\item Find the countries of Danube and where available their english labels.
\begin{lstlisting}[frame=single]
select distinct ?label
{

   dbr:Danube dbo:country ?country .
   ?country rdfs:label ?label .
   FILTER (lang(?label) = 'en')
} 
\end{lstlisting}


\item Find the source countries of Nile and where available their french labels.
\begin{lstlisting}[frame=single]
select distinct ?label
{

   dbr:Nile dbo:sourceCountry ?country .
   ?country rdfs:label ?label .
   FILTER (lang(?label) = 'fr')
} 
\end{lstlisting}

\item Find the homepage of Bob Marley
\begin{lstlisting}[frame=single]
select distinct ?y
{
   ?r foaf:name "Bob Marley"@en .
   ?r foaf:isPrimaryTopicOf ?y .

} 
\end{lstlisting}

\item Find all people influenced by Alan Turing
\begin{lstlisting}[frame=single]
select ?x
{
    ?x dbo:influencedBy dbr:Alan_Turing .
}
\end{lstlisting}

\item Find all people influenced by someone influenced by Alan Turing
\begin{lstlisting}[frame=single]
select distinct ?y
{
    ?x dbo:influencedBy dbr:Alan_Turing .
    ?y dbo:influencedBy ?x .
}
\end{lstlisting}

\item Find all people that were doctoral students of Alan Turing or influenced by him
\begin{lstlisting}[frame=single]
select distinct ?x
{
    {
        ?x dbo:influencedBy dbr:Alan_Turing .
    }
    union
    {
        ?x dbo:doctoralAdvisor dbr:Alan_Turing .
    }
}
\end{lstlisting}

\item Find all people that are connected to Alan Turing via the influenced by relationship
\begin{lstlisting}[frame=single]
select distinct ?x
{
    {
        ?x dbo:influencedBy+ dbr:Alan_Turing .
    }
    union
    {
        dbr:Alan_Turing dbo:influencedBy+ ?x .
    }
}
\end{lstlisting}

\item Give the list of the names of people that are linked via the doctoral student relationship with someone that has been a doctoral student of Alan Turing
\begin{lstlisting}[frame=single]
select distinct ?name
{
    ?x foaf:name ?name .
    ?x dbo:doctoralAdvisor+ dbr:Alan_Turing .
}
\end{lstlisting}

\item Give the names of all the persons that are connected to Alan Turing
\begin{lstlisting}[frame=single]
select distinct ?name  
{
     ?x foaf:name ?name .
     {
         ?x ?conn dbr:Alan_Turing .
         ?x a foaf:Person .
     } UNION {
        ?y ?conn dbr:Alan_Turing .
        ?y a foaf:Person .
        ?x ?conn ?y .
     }
}

\end{lstlisting}

\item Give the number of institutions Alan Turing worked for.
\begin{lstlisting}[frame=single]
select count(distinct ?x)
{
    dbr:Alan_Turing dbp:workInstitutions ?x .
    ?x ?y ?z .
}
\end{lstlisting}

\item Give the list of fields of Alan Turing except those that are also the fields of Alonzo Church
\begin{lstlisting}[frame=single]
select distinct ?x
{
    dbr:Alan_Turing dbo:field ?x .
    filter not exists 
    {
        dbr:Alonzo_Church dbo:field ?y .
        filter (?x == ?y)
    }
}
\end{lstlisting}

\item The list of scientists in the field of Logic ordered according to the the number of their doctoral students.
\begin{lstlisting}[frame=single]
select ?sci (count(distinct ?stu) as ?c)
{
    ?sci dbo:field dbr:Logic .
    ?stu dbo:doctoralAdvisor ?sci .
}
order by desc(?c) 
\end{lstlisting}

\item Give the most popular scientists with an average age of doctoral students above 30 years.
\begin{lstlisting}[frame=single]
select ?sci 
{
    ?inf dbo:influencedBy ?sci .
    ?stu dbo:doctoralAdvisor ?sci .
    ?stu dbo:birthDate ?birthdate ;
         dbo:deathDate ?deathdate .
    FILTER (datatype(?birthdate) = xsd:date && datatype(?deathdate) = xsd:date )
    bind( year(?deathdate) - year(?birthdate) - if(month(?deathdate)<month(?birthdate) || (month(?deathdate)=month(?birthdate) && day(?deathdate)<day(?birthdate)),1,0) as ?age )

}
group by ?sci
having (avg(?age) > 30)
order by desc(count(distinct ?inf))
limit 10
\end{lstlisting}
\end{enumerate}
\end{document}
