\documentclass{article}
\usepackage[utf8]{inputenc}
\usepackage{listings}
\usepackage[most]{tcolorbox}

\newtcblisting{commandshell}{colback=black,colupper=white,colframe=yellow!75!black,
listing only,listing options={language=sh, breaklines=true,aboveskip=0pt, belowskip=0pt},
every listing line={\small\ttfamily\bfseries{[oracle@tp-shchai]\$} }}

\newtcblisting{sqlshell}{colback=black,colupper=white,colframe=yellow!75!black,
listing only,listing options={language=SQL, breaklines=true, aboveskip=0pt, belowskip=0pt},
every listing line={\small\ttfamily\bfseries{SQL> }}}

\newtcblisting{messageshell}{colback=black,colupper=white,colframe=yellow!75!black,
listing only,listing options={language={}, basicstyle=\small\ttfamily, breaklines=true,aboveskip=0pt, belowskip=0pt},
every listing line={}}

\title{Database Management Experiment Report \#3}
\author{Shitong CHAI}
\date{\vspace{-5ex}}
\begin{document}
\maketitle

%%%%%%%%%%%%%%% section 1
\section{Data files and tablespaces}


\begin{enumerate}
%%%%%%%1
\item{Display tablepsaces}
\begin{sqlshell}
select tablespace_name from dba_tablespaces;
\end{sqlshell}
\begin{messageshell}
TABLESPACE_NAME
------------------------------
SYSTEM
SYSAUX
UNDOTBS1
TEMP
USERS

\end{messageshell}

%%%%%%%2
\item{Display tablespaces with associated data files}
\begin{sqlshell}
select dba_data_files.tablespace_name, dba_data_files.file_name from dba_data_files join dba_tablespaces on dba_tablespaces.tablespace_name = dba_data_files.tablespace_name;
\end{sqlshell}
\begin{messageshell}

TABLESPACE_NAME
------------------------------
FILE_NAME
--------------------------------------------------------------------------------
USERS
/opt/oracle/oradata/ORCLCDB/users01.dbf

UNDOTBS1
/opt/oracle/oradata/ORCLCDB/undotbs01.dbf

SYSTEM
/opt/oracle/oradata/ORCLCDB/system01.dbf


TABLESPACE_NAME
------------------------------
FILE_NAME
--------------------------------------------------------------------------------
SYSAUX
/opt/oracle/oradata/ORCLCDB/sysaux01.dbf

\end{messageshell}

%%%%%%%3
\item{Display default\_tablespaces for each user ( you can see dba\_users).}
\begin{sqlshell}
select username,default_tablespace from dba_users;
\end{sqlshell}
\begin{messageshell}
USERNAME
--------------------------------------------------------------------------------
DEFAULT_TABLESPACE
------------------------------
SYS
SYSTEM

SYSTEM
SYSTEM

XS\$NULL
SYSTEM


USERNAME
--------------------------------------------------------------------------------
DEFAULT_TABLESPACE
------------------------------
OJVMSYS
SYSTEM

LBACSYS
SYSTEM

OUTLN
SYSTEM


USERNAME
--------------------------------------------------------------------------------
DEFAULT_TABLESPACE
------------------------------
SYS\$UMF
SYSTEM

DBSNMP
SYSAUX

APPQOSSYS
SYSAUX


USERNAME
--------------------------------------------------------------------------------
DEFAULT_TABLESPACE
------------------------------
DBSFWUSER
SYSAUX

GGSYS
SYSAUX

ANONYMOUS
SYSAUX


USERNAME
--------------------------------------------------------------------------------
DEFAULT_TABLESPACE
------------------------------
CTXSYS
SYSAUX

DVF
SYSAUX

SI_INFORMTN_SCHEMA
SYSAUX


USERNAME
--------------------------------------------------------------------------------
DEFAULT_TABLESPACE
------------------------------
DVSYS
SYSAUX

GSMADMIN_INTERNAL
SYSAUX

ORDPLUGINS
SYSAUX


USERNAME
--------------------------------------------------------------------------------
DEFAULT_TABLESPACE
------------------------------
MDSYS
SYSAUX

OLAPSYS
SYSAUX

ORDDATA
SYSAUX


USERNAME
--------------------------------------------------------------------------------
DEFAULT_TABLESPACE
------------------------------
XDB
SYSAUX

WMSYS
SYSAUX

ORDSYS
SYSAUX


USERNAME
--------------------------------------------------------------------------------
DEFAULT_TABLESPACE
------------------------------
GSMCATUSER
USERS

MDDATA
USERS

SYSBACKUP
USERS


USERNAME
--------------------------------------------------------------------------------
DEFAULT_TABLESPACE
------------------------------
REMOTE_SCHEDULER_AGENT
USERS

GSMUSER
USERS

SYSRAC
USERS


USERNAME
--------------------------------------------------------------------------------
DEFAULT_TABLESPACE
------------------------------
AUDSYS
USERS

DIP
USERS

SYSKM
USERS


USERNAME
--------------------------------------------------------------------------------
DEFAULT_TABLESPACE
------------------------------
ORACLE_OCM
USERS

SCOTT
USERS

SYSDG
USERS


36 rows selected.

\end{messageshell}

%%%%%%%4
\item{Allocate 400 kbytes additional space to the tablespace USERS and verify the result.}
% https://docs.oracle.com/cd/B28359_01/server.111/b28310/dfiles003.htm#ADMIN11425

Use select to get original tablespace size:
\begin{sqlshell}
select tablespace_name, bytes /1024  " size ( KB )" from dba_data_files; 
\end{sqlshell}
\begin{messageshell}
TABLESPACE_NAME         size ( KB )
------------------------------ ------------
USERS                       5120
UNDOTBS1                 317440
SYSTEM                     890880
SYSAUX                    1341440


\end{messageshell}

Allocate 400K more space to USERS:
\begin{sqlshell}
alter database datafile '/opt/oracle/oradata/orclcdb/users01.dbf' resize 5520k;
\end{sqlshell}
\begin{messageshell}

Database altered.

\end{messageshell}

Use select to verify the result:
\begin{sqlshell}
select tablespace_name, bytes /1024  " size ( KB )" from dba_data_files; %to check size
\end{sqlshell}
\begin{messageshell}
TABLESPACE_NAME         size ( KB )
------------------------------ ------------
USERS                       5520
UNDOTBS1                 317440
SYSTEM                     890880
SYSAUX                    1341440

\end{messageshell}

%%%%%%%%%%%%5
\item{Create a tablespace DATA1TBS with the datafile /home/oracle/DISK2/data01.dbf (initial file size : 2M, maximum file size : 5M). Activate the 500K automatic extension if additional extents are required.}
\begin{commandshell}
cd /home/oracle 
mkdir DISK2
\end{commandshell}
\begin{sqlshell}
create tablespace DATA1TBS datafile '/home/oracle/DISK2/data01.dbf' size 2m autoextend on next 500k maxsize 5m;
\end{sqlshell}
\begin{messageshell}

Tablespace created.

\end{messageshell}

%%%%%%%%%%%%6
\item{Move the tablespace DATA1TBS in the directoy DISK4 }
\begin{commandshell}
mkdir /home/oracle/DISK4
cp /home/oracle/DISK2/data01.dbf /home/oracle/DISK4/data01.dbf
\end{commandshell}
\begin{sqlshell}
alter tablespace DATA1TBS offline normal;
\end{sqlshell}
\begin{messageshell}

Tablespace altered.
\end{messageshell}

\begin{sqlshell}
alter tablespace DATA1TBS rename datafile '/home/oracle/DISK2/data01.dbf' to '/home/oracle/DISK4/data01.dbf';
\end{sqlshell}
\begin{messageshell}

Tablespace altered.
\end{messageshell}

%%%%%%%%%%%%7
\item{Create a tablespace RONLY. create a table in RONLY. Put the tablespace in read-only mode. Try to create another table in RONLY tablespace.}
\begin{sqlshell}
create tablespace RONLY datafile '/home/oracle/DISK2/data02.dbf' size 2m autoextend on next 500k maxsize 5m;
create table tbl (startdate DATE, enddate DATE, class CHAR(20)) tablespace RONLY;
alter tablespace RONLY read only;
\end{sqlshell}
\begin{messageshell}

Tablespace created.
Table created.
Tablespace altered.
\end{messageshell}
\begin{sqlshell}
create table tbl2 (startdate DATE, enddate DATE, class CHAR(20)) tablespace RONLY;
\end{sqlshell}
\begin{messageshell}
create table tbl2 (startdate DATE, enddate DATE, class CHAR(20)) tablespace RONLY
*
ERROR at line 1:
ORA-01647: tablespace 'RONLY' is read-only, cannot allocate space in it

\end{messageshell}
%%%%%%%%%%%8
\item{Delete the tablespace RONLY and check the operation. (you must physically delete the associated files to the tablespace by yourserf.)}
\begin{sqlshell}
drop tablespace RONLY including contents;
\end{sqlshell}
\begin{messageshell}

Tablespace dropped.

\end{messageshell}
\begin{commandshell}
rm data02.dbf
\end{commandshell}

%%%%%%%%%%%9
\item{Create a temporary tablespace TEMPY. Check the operation.}
\begin{sqlshell}
create temporary tablespace TEMPY tempfile 'temp01.dbf' size 5m autoextend on;
\end{sqlshell}
\begin{messageshell}

Tablespace created.

\end{messageshell}
\begin{sqlshell}
select tablespace_name from dba_tablespaces;
\end{sqlshell}
\begin{messageshell}

TABLESPACE_NAME
------------------------------
SYSTEM
SYSAUX
UNDOTBS1
TEMP
USERS
DATA1TBS
TEMPY

7 rows selected.
\end{messageshell}
\end{enumerate}

%%%%%%%%%%%%%%% section 1
\section{Segments }
\begin{enumerate}
\item{Identify the different types of segments in the database.}
\begin{sqlshell}
select distinct segment_type from dba_segments;
\end{sqlshell}
\begin{messageshell}
SEGMENT_TYPE
------------------
INDEX
CLUSTER
TABLE PARTITION
LOBINDEX
TABLE SUBPARTITION
SYSTEM STATISTICS
LOBSEGMENT
INDEX PARTITION
ROLLBACK
TABLE
LOB PARTITION

SEGMENT_TYPE
------------------
NESTED TABLE
TYPE2 UNDO
DEFERRED ROLLBACK

14 rows selected.
\end{messageshell}

\item{What are the files containing the data of the EMP table of scott?}

\begin{sqlshell}
select b.file_name from dba_extents a, dba_data_files b where a.file_id=b.file_id and segment_name='EMP' and owner='SCOTT';
\end{sqlshell}
\begin{messageshell}

FILE_NAME
--------------------------------------------------------------------------------
/opt/oracle/oradata/ORCLCDB/users01.dbf

\end{messageshell}
\item{Display the free space by tablespace. }
\begin{messageshell}
SQL> select tablespace_name, round(sum(bytes)/1024/1024 ,2) FreeMB
            from dba_free_space group by tablespace_name;
\end{messageshell}
\begin{messageshell}
TABLESPACE_NAME            FREEMB
------------------------------ ----------
SYSTEM                     5.38
SYSAUX                     89.5
UNDOTBS1               292.75
USERS                     4.19

\end{messageshell}
\end{enumerate}

\end{document}
